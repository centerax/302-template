\documentclass[12pt,a4paper,oneside]{book}
\usepackage[utf8]{inputenc}
\usepackage[spanish]{babel}
\usepackage[toc,page]{appendix}
\usepackage{longtable}
\usepackage{amsmath}
\usepackage{amsfonts}
\usepackage{amssymb}
\usepackage{listingsutf8}
\usepackage{mathptmx}%fuente Times
\usepackage{sectsty}
\allsectionsfont{\fontfamily{phv}\selectfont}
\chapterfont{\fontfamily{phv}\selectfont}
\usepackage{color, soul}
\usepackage{paracol}
\usepackage{lipsum}
\usepackage[T1]{fontenc}
\usepackage{graphicx}
\usepackage{float}
\graphicspath{ {Imagenes/} }
\definecolor{Azul}{rgb}{0,0,1}
\definecolor{Negro}{rgb}{0,0,0}
\definecolor{Rojo}{rgb}{1,0,0}
\definecolor{Naranja}{rgb}{1,0.75,0}
\definecolor{Verde}{rgb}{0,0.30,0}
\definecolor{Celeste}{rgb}{0,0.75,0.75}
\definecolor{Azul}{rgb}{0,0,1}
\definecolor{Violeta}{rgb}{0.5,0,0.75}

%%%%%%%%%%%%% Macros para formato 302 %%%%%%%%%%%%%%%%%%%

%https://www.ort.edu.uy/reglamentos/normas-especificas-para-la-presentacion-de-trabajos-finales-de-carrera-facultad-de-ingenieria-excepto-biotecnologia__documento-302.pdf

%nombre del archivo: <Fecha  de  entrega  en  formato  AAAAMMDD>-<código  de  la  carrera>-<Número  de  Estudiante  1>-<Número   de   Estudiante   2>....pdf

% Metadata para el PDF a generar 
\usepackage[unicode,
            pdftex]{hyperref}
\hypersetup{
 pdfauthor={PONER LOS NOMBRES DE LOS AUTORES},
 pdftitle={PONER EL TITULO},
 pdfsubject={PONER EL SUBTITULO},
 pdfkeywords={PONER PALABRAS CLAVE},
 pdflang={es-UY},
 pdfproducer={texmaker}
 }
\hypersetup{
	colorlinks=true,
	urlcolor=blue,
	linkcolor=blue
}
% Numeración y formato de páginas
\usepackage{fancyhdr}
\pagestyle{fancy}
\fancyhf{}
%\fancyhead[lo,le]{\nouppercase{\rightmark}}
\fancyfoot[RO]{\thepage}

\fancypagestyle{plain}{
  \fancyhf{} 
  \fancyfoot[RO]{\thepage}
  \renewcommand{\headrulewidth}{0pt}}
\renewcommand{\headrulewidth}{0pt}


% Indice (estos parámetros se pueden cambiar)
\setcounter{secnumdepth}{3} %para que ponga 1.1.1.1 en subsubsecciones
\setcounter{tocdepth}{3} % para que ponga subsubsecciones en el indice

%% Para incluir imagenes
\usepackage{graphicx}


% Para incluir tablas en español
\renewcommand\tablename{\bfseries Tabla}

% Para incluir listings de código
\renewcommand{\lstlistingname}{\bfseries C\'odigo}

% Para agregar la bibliografía al indice
\usepackage[nottoc,numbib]{tocbibind}


%%%%%%%%%%%%%%%%%%%%%%%%%%%%%%%%%%%%%%%%%%%%%%%%%%%%%%

% Generación del índice
\makeindex

% Empieza el documento
\begin{document}

% Portada (completar con los datos)
\vspace*{\fill}

\begin{center}

\begin{Large}
\textbf{Universidad ORT Uruguay}

\textbf{Facultad de Ingeniería}
\vspace{5cm}
\end{Large}

\begin{huge}
[Título del trabajo]
\end{huge} 

\vspace{1cm}

Entregado como requisito para la obtención del título de [nombre del título]
\vspace{2cm}

\begin{Large}
[Nombre Apellido - Número de estudiante]\\

[Nombre Apellido 2 - Número de estudiante 2]
\vspace{2cm}

Tutor: [Nombre Apellido]
\vspace{2cm}
\end{Large}

\begin{large}
[Año de la entrega]
\end{large}

\end{center}
\vspace*{\fill}

\thispagestyle{empty}
\newpage




\chapter*{Declaración de autoría}




Nosotros, [nombres de los autores], declaramos que el trabajo que se presenta en esta obra es de nuestra
propia mano. Podemos asegurar que:
\begin{list}{$\bullet$}{}
	\item La obra fue producida en su totalidad mientras realizábamos [nombre de la actividad curricular que origina la obra];
	\item Cuando hemos consultado el trabajo publicado por otros, lo hemos atribuido con claridad;
	\item Cuando hemos citado obras de otros, hemos indicado las fuentes. Con excepción de estas citas, la obra es enteramente nuestra;
	\item En la obra, hemos acusado recibo de las ayudas recibidas;
	\item Cuando la obra se basa en trabajo realizado conjuntamente con otros, hemos explicado claramente qué fue contribuido por otros, y qué fue contribuido por nosotros;
	\item Ninguna parte de este trabajo ha sido publicada previamente a su entrega, excepto donde se han realizado las aclaraciones correspondientes.
\end{list}

 
\vspace{2cm}


%Incluir las firmas escaneadas en el mismo directorio, con los nombres signature1.jpg, signature2.jpg,.. y descomentar las líneas de abajo.\\

%\includegraphics[scale=0.4]{signature1.jpg} \hfill %\includegraphics[scale=0.4]{signature2.jpg} \hfill 

\begin{minipage}[b]{0.33333\textwidth}
\centering
[Firma del autor] \par
insertada gráficamente
[aclaración de firma]
[Fecha del día]
\end{minipage}%
\begin{minipage}[b]{0.33333\textwidth}
\centering
[Firma del autor] \par
insertada gráficamente
[aclaración de firma]
[Fecha del día]
\end{minipage}%
\begin{minipage}[b]{0.33333\textwidth}
\centering
[Firma del autor] \par
insertada gráficamente
[aclaración de firma]
[Fecha del día]
\end{minipage}%

\chapter*{Dedicatoria - opcional}
No es obligatorio incluir esta sección.

\chapter*{Agradecimientos - opcional}
No  es  obligatorio  incluir  esta  sección.  Se  trata  de  un  breve  reconocimiento  a  personas  o  instituciones que de diversas maneras han ayudado en la elaboración del trabajo. En  caso  de  incluir  agradecimientos,  asegúrese  de  usar  los  nombres  correctos  y  completos  de  las  organizaciones y las personas que cite aquí.

\chapter*{Abstract}
Consiste  en  un  resumen  del  contenido  del  trabajo,  que  se  usa  para  difusión  y  para  que  el  lector  potencial sepa en qué consiste el trabajo sin necesidad de leerlo completamente. Puede tener una extensión máxima de 400 palabras. Debe existir coherencia entre el contenido del trabajo final y el  abstract.  Ver  documento  306  (Orientación  para  títulos,  resúmenes  o  abstract  e  informes  de  corrección de trabajos finales de carrera).

\chapter*{Palabras clave}
Conjunto  de  palabras  que  están  directamente  relacionadas  con  el  contenido  de  la  obra,  que  además deberán ser incluidas en las propiedades del archivo PDF para facilitar que el documento sea  indexado  por  los  buscadores.  Deberían  ser  suficientemente  específicas,  evitando  la  excesiva  generalidad.

%el siguiente comando genera el índice (no tocar)
\tableofcontents

\chapter{Cuerpo}
\lipsum[2]


\bibliographystyle{IEEEtran}
%%bibliography
%% No borrar estas dos líneas. Revisar de tener los archivos IEEETran y biblio.bib en la misma carpeta
%% si no están y no anda, correr el compilador de latex varias veces. Primero el bibtex, luego pdflatex, luego pdflatex otra vez.
\bibliography{biblio}

\appendix %A partir de acá los capítulos se enumerarán A, B, C, etc

\chapter{Nombre del Apéndice 1}
\cite{girard1989}
\lipsum[1]	
\section{Una Sección}
\lipsum[2-3]
\subsection{Una Subsección}
\lipsum[4-6]

\end{document}

